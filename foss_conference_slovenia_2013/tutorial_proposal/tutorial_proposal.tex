%%%%%%%%%%%%%%%%%%%%%%%%%% author.tex %%%%%%%%%%%%%%%%%%%%%%%%%
%
% sample root file for your contribution to an IFIP volume
% published at Springer
%
% Use this file as a template for your own input.
%
%%%%%%%%%%%%%%%%%%%%%%%% Springer-Verlag %%%%%%%%%%%%%%%%%%%%%%%%%%


% RECOMMENDED %%%%%%%%%%%%%%%%%%%%%%%%%%%%%%%%%%%%%%%%%%%%%%%%%%%
\documentclass[ifip]{svmult}

% choose options for [] as required from the list
% in the Reference Guide, Sect. 2.2
\usepackage{epsfig} 
\usepackage{makeidx}         % allows index generation
\usepackage{graphicx}        % standard LaTeX graphics tool
                             % when including figure files
\usepackage{multicol}        % used for the two-column index
\usepackage[bottom]{footmisc}% places footnotes at page bottom
% etc.
% see the list of further useful packages
% in the Reference Guide, Sects. 2.3, 3.1-3.3

\begin{document}

\title{Tutorial: Open Source Scientific Computing}
\author{Amit Saha}
\institute{Red Hat, Inc, 193 North Quay, Brisbane, Australia.
\textit{asaha@redhat.com}}

% Use the package "url.sty" to avoid
% problems with special characters
% used in your e-mail or web address
%
\maketitle

In this tutorial, the audience will learn how they can adopt Open
Source tools and methodologies in their research.

\section{Why Open Source?}
\label{sec:1}
The Open Source ecosystem has excellent tools for the scientific
community. Over the years, thanks to contributions by the various
communities, these tools have matured and have been adopted for
scientific research in various fields. They are now more than capable
alternatives to expensive proprietary software. Besides their economic
advantages, they also bring with them the freedom to be used, tinkered
with and redistributed with liberal licenses.  

Adopting Open Source methodologies in research enables sharing of tools and data with an
appropriate license. This serves the dual purpose of open-ness of the research 
and also protects the intellectual rights from being abused. 

The next section outlines the various software tools that will be
addressed in this tutorial.

\section{Open Source Tools}

\paragraph{Operating System:}

The \textit{Fedora Scientific Spin} is a freely available Linux based operating system,
exclusively targeting computer users conducting scientific research. It has a
plethora of tools already installed which makes it easy to get started
with, out of the box. Other choices include \textit{Scientific Linux}.

\paragraph{Numerical Computing tools:}

\textit{GNU Octave}, \textit{Scilab} and \textit{Sage} aim to provide
an all-in-one scientific software package with support for various numerical and
scientific computing requirements.

\paragraph{C/C++ Libraries:}

The \textit{GNU Scientific Library} is a library for C/C++
implementing an extensive list of mathematical routines. 

The \textit{GNU multi precision} library adds support for
multi-precision arithmetic for C/C++ programs.

\paragraph{Python for Scientific Computing:}

The \textit{Python} programming language combines rapid prototyping
features with an easy syntax. Third party libraries such
as SciPy with a gentle learning curve makes Python a first class
choice for scientific computing for beginners and veterans alike.

\paragraph{R for Data analysis:}

\textit{R} is a programming language and environment for statistical
computing. It provides a wide variety of modeling and data analysis
techniques and has become a de-facto standard for analyzing data and
making statistical inferences based on them.

\paragraph{Parallel and Distributed Computing:}

Certain areas of scientific research demand the availability of high
performance computing power. Traditional open source solutions such as
\textit{Open MPI} and modern cloud computing based solutions make it
easy to harness the collective power of a lab of low cost
computers. These tools are supported by libraries for popular
programming languages to make them easy to use.

\paragraph{Drawing and Plotting:}

Plotting tools such as \textit{Gnu plot} and drawing tools such as
\textit{Xfig} and \textit{Dia} allow creation of high quality graphs
and figures and exporting them to \textit{encapsulated postscript}
(eps) to be used with the LaTex typesetting system.


\section{Open Source Methodologies}

\paragraph{Open Source Licensing:}

Computer programs -- either developed from scratch or built upon someone
else's work form the cornerstone of computational scientific
research. Any researcher or interested reader can easily verify and
better undestand a research article's results simply by running those
programs. Hence, it is imperative that such programs are made publicly
available. Sharing programs using a suitable Open Source license
protects the interests and rights of the creator of the program, while
also giving anyone else the right to read, enhance and improve those
programs. This goes a long way in increasing transparency of the whole
scientific process.

\paragraph{Version Control:}

Version control enables inter and intra group research
collaboration. The benefits of version control are not only limited to
computer programs, but any other data -- research papers, data sets,
figures and others. 

\section{Conclusion}

Scientific computing stands to benefit highly if it adopts Open Source
computing tools and methodologies. Ultimately, it will result in a
more open scientific research community and hence enable and foster innovation on
a larger scale than it is.

\end{document}





